
\section*{Abstract}  
We present results from an experiment which measures the lifetime of the muon. This measurement was done by detecting positively charged muons ($\mu^+$) approaching the apparatus from the atmosphere, stopping them when they are incident on a thick aluminum slab, and measuring the time it takes to detect a resultant positron ($e^+$) from the muon's decay process. The distribution of delay times between muon ($\mu^+$) and positron ($e^+$) detection provide an understanding of the muon lifetime. We found that the average time difference between $\mu^+$ detection and $e^+$ detection was $\tau_{avg} = 2.216679 \pm 0.013 \times 10^{-6}$ s. Therefore the muon lifetime predicted by this experiment was within $10^{-8}$s of the expected lifetime \mbox{$\tau_\mu = 2.196981(22) \times 10^{-6}$} s \cite{ber}\cite{pat}, and with a high confidence 
