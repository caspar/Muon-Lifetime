\section*{Introduction}

In this experiment, we measured the lifetime of cosmic muons by measuring the interval of time it took to decay. We did this by operating an apparatus that stopped the muons from the atmosphere using a slab of aluminum. When the muons are stopped, the apparatus measures the interval of time between the initial detection of the muon, and the detection of the resulting positron after the decay. We used the distribution of intervals of time in order to fit a decaying exponential and recover the missing parameter $\tau$, for the $\mu^{+}$ lifetime. 

The objective of this experiment was to statistically determine the average interval of time it takes for muons from the upper atmosphere to decay via the following decay process
\begin{align}
    \mu^+ \rightarrow e^+ + \nu_e + \Bar{\nu}_\mu.
\end{align}

Through the use of a photomultiplier and time-to-amplitude converter, we were able to convert the signal from the scintillation detector into an electronic voltage, representing the presence of a time delay due to muon decay. We then fitted a decaying exponential for the parameter $\tau$, the muon lifetime:
\begin{align}
    N(t)= \lambda e^{-\lambda t},
\end{align}
where $\lambda = 1 / \tau$.
